\documentclass[a4paper,12pt]{article}
\usepackage[T2A]{fontenc}
\usepackage[utf8]{inputenc}	
\usepackage[english,russian]{babel}	
\usepackage{amsmath,amsfonts,amssymb,amsthm,mathtools} 
\usepackage[left=16mm,  top=15mm,  right=16mm,  bottom=20mm, nohead, nofoot]{geometry}
\usepackage{graphicx}
\graphicspath{{pictures/}}
\DeclareGraphicsExtensions{.pdf,.png,.jpg}
\usepackage{wasysym}
\usepackage{tikz}
\usepackage[unicode, pdftex]{hyperref}

\begin{document} 

\begin{center}
\Large{\textbf{Домашняя работа 1}}

\Large{\textbf{Богданов Александр Б05-003}}
\end{center}

\section{Задача}

Дано:
    \[f(X) = \det (X^{-1} + A)\]
Известно:
    \[d(\det X) = \langle \det (X)X^{-T}, dX \rangle\]
    \[d(X^{-1}) = - X^{-1} d(X) X^{-1}\]
Решение:
    \[df(X) = \langle \det (X^{-1} + A)(X^{-1} + A)^{-T}, d(X^{-1} + A) \rangle = - \det (X^{-1} + A) \langle (X^{-1} + A)^{-T}, X^{-1} d(X) X^{-1} \rangle = \]
    \[= - \det (X^{-1} + A) \langle X^{-T}(X^{-1} + A)^{-T}X^{-T}, dX  \rangle = - \det (X^{-1} + A) \langle (X(X^{-1} + A)X)^{-T}, dX  \rangle = \]
    \[= - \det (X^{-1} + A) \langle (X + XAX)^{-T}, dX  \rangle \]
Тогда:
    \[\nabla_X f = - \det (X^{-1} + A) (X + XAX)^{-T}\]
\textbf{Ответ:}
    \[\nabla_X f = - \det (X^{-1} + A) (X + XAX)^{-T}\]

\section{Задача}

\subsection{Пункт a)}

Дано:
    \[f(t) = \|(A + tI_n)^{-1}b\|_2\]
Известно:
    \[d(\|x\|_2) = \frac{\langle x, dx \rangle}{\|x\|_2}\]
    \[d(X^{-1}) = - X^{-1} d(X) X^{-1}\]
Решение (первая часть):
    \[df(t) = \frac{\langle (A + tI_n)^{-1}b, d((A + tI_n)^{-1}b) \rangle}{\|(A + tI_n)^{-1}b\|_2} = - \frac{\langle (A + tI_n)^{-1}b, (A + tI_n)^{-1}dt(A + tI_n)^{-1}b \rangle}{\|(A + tI_n)^{-1}b\|_2} = \]
    \[= - \frac{b^T(A + tI_n)^{-T}(A + tI_n)^{-2}b}{\|(A + tI_n)^{-1}b\|_2} dt = - \frac{b^T(A + tI_n)^{-3}b}{\|(A + tI_n)^{-1}b\|_2} dt\]
Тогда:
    \[\nabla f = - \frac{b^T(A + tI_n)^{-3}b}{\|(A + tI_n)^{-1}b\|_2}\]
Решение (вторая часть):
    \[d^2f(t) = - \frac{\|(A + tI_n)^{-1}b\|_2 d(b^T(A + tI_n)^{-3}b) - d(\|(A + tI_n)^{-1}b\|_2)(b^T(A + tI_n)^{-3}b)}{\|(A + tI_n)^{-1}b\|_2^2} dt = \]
    \[= - \frac{b^Td(A + tI_n)^{-3}b}{\|(A + tI_n)^{-1}b\|_2} dt + \frac{d(\|(A + tI_n)^{-1}b\|_2)(b^T(A + tI_n)^{-3}b)}{\|(A + tI_n)^{-1}b\|_2^2} dt = \]
    \[= 3 \frac{b^T(A + tI_n)^{-4}b}{\|(A + tI_n)^{-1}b\|_2} (dt)^2 - \frac{(b^T(A + tI_n)^{-3}b)^2}{\|(A + tI_n)^{-1}b\|_2^3} (dt)^2\]
Тогда:
    \[\nabla^2f = - \frac{(b^T(A + tI_n)^{-3}b)^2}{\|(A + tI_n)^{-1}b\|_2^3} + 3 \frac{b^T(A + tI_n)^{-4}b}{\|(A + tI_n)^{-1}b\|_2}\]
\textbf{Ответ:}
    \[\nabla f = - \frac{b^T(A + tI_n)^{-3}b}{\|(A + tI_n)^{-1}b\|_2}\]
    \[\nabla^2f = - \frac{(b^T(A + tI_n)^{-3}b)^2}{\|(A + tI_n)^{-1}b\|_2^3} + 3 \frac{b^T(A + tI_n)^{-4}b}{\|(A + tI_n)^{-1}b\|_2}\]
    
\subsection{Пункт b)}

Дано:
    \[f(t) = \frac{1}{2} \|xx^T - A\|_F^2\]
Известно:
    \[d\langle Ax, x \rangle = \langle (A + A^T)x, dx \rangle \]
Решение (первая часть):
    \[df(t) = \frac{1}{2} d\langle xx^T - A, xx^T - A \rangle = \langle xx^T - A, d(xx^T - A) \rangle = \langle xx^T - A, d(x)x^T + xd(x)^T \rangle =\]
    \[= \langle (xx^T - A)x, dx \rangle + \langle (xx^T - A)x, dx \rangle = \langle 2(xx^T - A)x, dx \rangle\]
Тогда:
    \[\nabla f = 2(xx^T - A)x\]
Решение (вторая часть):
    \[d^2f(t) = \langle d(2(xx^T - A)x), dx_1 \rangle = \langle 2(d(xx^Tx) - Adx_2), dx_1 \rangle = \langle 2(dx_2(x^Tx) + x d\langle x, x \rangle - Adx_2), dx_1 \rangle =\]
    \[= \langle 2(dx_2(x^Tx) + 2xx^Tdx_2 - Adx_2), dx_1 \rangle = \langle 2x^TxI + 4xx^T - 2A)dx_2, dx_1 \rangle = \langle (2x^TxI + 4xx^T - 2A)dx_1, dx_2 \rangle\]
Тогда:
    \[\nabla^2f = 2x^TxI + 4xx^T - 2A\]
\textbf{Ответ:}
    \[\nabla f = 2(xx^T - A)x\]
    \[\nabla^2f = 2x^TxI + 4xx^T - 2A\]

\section{Задача}

Воспользуемся алгоритмом backpropagation:
    \[dL(x) = \langle \nabla_x L, dx \rangle\]
    \[dx(A,b) = d(A^{-1} b) = -A^{-1} dA A^{-1} b + A^{-1} db\]
    \[dL(A,b) = - \langle \nabla_x L, A^{-1} dA A^{-1} b \rangle + \langle \nabla_x L, A^{-1} db \rangle = - \langle A^{-T} \nabla_x L b^T A^{-T}, dA \rangle + \langle A^{-T} \nabla_x L, db \rangle\]
Так как в стандартном виде $dL = \langle \nabla_A L, dA \rangle + \langle \nabla_b L, db \rangle$ получаем:
    \[\nabla_A L = -A^{-T} \nabla_x L b^T A^ {-T}\]
    \[\nabla_b L = A^{-T} \nabla_x L\]
\textbf{Ответ:}
    \[\nabla_A L = -A^{-T} \nabla_x L b^T A^{-T}\]
    \[\nabla_b L = A^{-T} \nabla_x L\]

\section{Задача}

\subsection{Пункт a)}

Дано:
    \[f(x) = \langle a, x \rangle - \ln (1 - \langle b, x \rangle),\ a, b \not = 0,\ E = \{x \in \mathbb{R}^n | \langle x, b \rangle < 1 \}\]
Решение:
    \[df(x) = \langle a, dx \rangle + \frac{1}{1 - \langle b, x \rangle} \langle b, dx \rangle = \left\langle a + \frac{b}{1 - \langle b, x \rangle}, dx \right\rangle\]
    \[\nabla f = a + \frac{b}{1 - \langle b, x \rangle} = 0 \Rightarrow a \text{ и } b \text{ должны быть линейно зависимыми, так как } a, b \not = 0\]
Тогда $b = \lambda a \Rightarrow 1 - \langle b, x \rangle + \lambda = 0 \Rightarrow \langle b, x \rangle = \lambda + 1 \Rightarrow$ для того, чтобы выполнялось условие принадлежности множеству $\lambda$ должна быть меньше нуля.\\
\textbf{Ответ:}
\begin{enumerate}
    \item Если $a$ и $b$ линейно независимы, то точек стационарности нет,
    \item Если $b = \lambda a$ и $\lambda \geq 0$, то точек стационарности нет на $E$,
    \item Если $b = \lambda a$ и $\lambda < 0$, то точки стационарности на $E$ это точки гиперплоскости $\langle b, x \rangle = \lambda + 1$.
\end{enumerate}

\subsection{Пункт b)}

Дано:
    \[f(x) = \langle c, x \rangle \exp{(-\langle Ax, x \rangle)},\ c \not = 0\]
Известно:
    \[d\langle Ax, x \rangle = \langle 2Ax, dx \rangle\ \text{для симметричной матрицы}\]
Решение:
    \[df(x) = \langle c, dx \rangle \exp{(-\langle Ax, x \rangle)} - \langle c, x \rangle \langle 2Ax, dx \rangle \exp{(-\langle Ax, x \rangle)} = \]
    \[= \langle (c - 2\langle c, x \rangle Ax) \exp{(-\langle Ax, x \rangle)}, dx \rangle\]
    \[\nabla f = (c - 2\langle c, x \rangle Ax) \exp{(-\langle Ax, x \rangle)} = 0\]
    \[2\langle c, x \rangle Ax = c\]
    \[\langle c, x \rangle x = \frac{1}{2}A^{-1}c\]
    \[\langle c, x \rangle^2 = \frac{1}{2}c^TA^{-1}c\]
    \[\langle c, x \rangle = \pm \sqrt{\frac{1}{2}c^TA^{-1}c},\ \text{можно, так как матрица положительно определенная}\]
    \[\frac{1}{2}A^{-1}c = \pm x \sqrt{\frac{1}{2}c^TA^{-1}c}\]
    \[x = \pm \frac{A^{-1}c}{\sqrt{2c^TA^{-1}c}}\]
\textbf{Ответ:}
    При любых параметрах ($c \not = 0$) существуют стационарные точки: $x = \pm \frac{A^{-1}c}{\sqrt{2c^TA^{-1}c}}$
    
\subsection{Пункт c)}

Дано:
    \[f(X) = \langle X^{-1}, I_n \rangle - \langle A, X \rangle\]
Известно:
    \[d(X^{-1}) = - X^{-1} d(X) X^{-1}\]
Решение:
    \[df(X) = f(X) = - \langle X^{-1} d(X) X^{-1}, I_n \rangle - \langle A, dX \rangle = - \langle dX, X^{-T}X^{-T} \rangle - \langle A, dX \rangle = \]
    \[= \langle - X^{-T}X^{-T} - A, dX \rangle\]
    \[\nabla f = - (XX)^{-T} - A = 0\]
    \[X^{-2} = - A\]
Если $A$ не отрицательно определенная матрица, то точек стационарности нет, так как слева положительно определенная матрица. Если $A$ отрицательно определенная матрица, то получаем:
    \[X^2 = - A^{-1}\]
Так как $A$ симметричная, то ее можно представить в виде: $A = Q^T \Lambda Q$, где $Q$ - ортогональная матрица, $\Lambda$ - диагональная матрица из собственных чисел. Тогда $- A^{-1} = Q^T (-\Lambda^{-1}) Q$ и следовательно $X = Q^T \sqrt{-\Lambda^{-1}} Q$
\textbf{Ответ:}
\begin{enumerate}
    \item Если $A$ не отрицательно определенная, то точек стационарности нет,
    \item Если $A$ отрицательно определенная, $X = Q^T \sqrt{-\Lambda^{-1}} Q$, где $A = Q^T \Lambda Q$, $Q$ - ортогональная матрица, $\Lambda$ - диагональная матрица из собственных чисел.
\end{enumerate}

\section{Задача}

Воспользуемся алгоритмом backpropagation:
    \[df(X) = \langle \nabla_X f, dX \rangle\]
    \[dX(\Lambda, Q) = d(Q^T \Lambda Q) = (dQ)^T \Lambda Q + Q^T d\Lambda Q + Q^T \Lambda dQ\]
    
    \[df(\Lambda, Q) = \langle \nabla_X f, (dQ)^T \Lambda Q \rangle + \langle \nabla_X f, Q^T d\Lambda Q \rangle + \langle \nabla_X f, Q^T \Lambda dQ \rangle = \]
    \[= \langle \Lambda Q (\nabla_X f)^T, dQ \rangle + \langle Q \nabla_X f Q^T, d\Lambda \rangle + \langle \Lambda Q \nabla_X f, dQ \rangle =\]
    \[= \langle Q \nabla_X f Q^T, d\Lambda \rangle + \langle \Lambda Q ((\nabla_X f)^T + \nabla_X f), dQ \rangle\]
Так как в стандартном виде $dL = \langle \nabla_Q f, dQ \rangle + \langle \nabla_{\Lambda} f, d\Lambda \rangle$ получаем:
    \[\nabla_Q f = \Lambda Q ((\nabla_X f)^T + \nabla_X f)\]
    \[\nabla_{\Lambda} f = Q \nabla_X f Q^T\]
    \[\nabla_Q f = 2 \Lambda Q \nabla_{\Lambda} f Q\]
    \[\nabla_X f = Q^T \nabla_{\Lambda} f Q\]
\textbf{Ответ:}
    \[\nabla_X f = Q^T \nabla_{\Lambda} f Q\]
    \[\text{При условии: } \nabla_Q f = 2 \Lambda Q \nabla_{\Lambda} f Q\]
\end{document}